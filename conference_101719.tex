\documentclass[conference]{IEEEtran}
\IEEEoverridecommandlockouts
% The preceding line is only needed to identify funding in the first footnote. If that is unneeded, please comment it out.
\usepackage{cite}
\usepackage{amsmath,amssymb,amsfonts}
\usepackage{algorithmic}
\usepackage{graphicx}
\usepackage{textcomp}
\usepackage{xcolor}
\usepackage{multirow} % Tables
\usepackage{booktabs}
\usepackage{caption}  % for controlling caption formatting
\usepackage{tabularx} % for controlling table width
\usepackage{url}

\def\BibTeX{{\rm B\kern-.05em{\sc i\kern-.025em b}\kern-.08em
    T\kern-.1667em\lower.7ex\hbox{E}\kern-.125emX}}

% Adjust caption font size and alignment
\captionsetup[table]{font=small, labelfont=bf}


\begin{document}

\title{Generating Movement Flow in 3D Jump'n Runs}

\author{\IEEEauthorblockN{Tobias Brandner}
\IEEEauthorblockA{\textit{Julius Maximilian University} \\
Würzburg, Germany \\
tobias.brandner\\@stud-mail.uni-wuerzburg.de}
\and
\IEEEauthorblockN{Marc Mußmann}
\IEEEauthorblockA{\textit{Julius Maximilian University} \\
Würzburg, Germany \\
marc.mussmann\\@stud-mail.uni-wuerzburg.de}
\and
\IEEEauthorblockN{Sebastian von Mammen}
\IEEEauthorblockA{\textit{Julius Maximilian University} \\
Würzburg, Germany \\
sebastian.von.mammen\\@uni-wuerzburg.de}
}

\maketitle

\begin{abstract}
    This research examines the flow of movement in 3D Jump'n Run games, focusing on the control and interaction of the player character in a simulated space. 
    The study compares various motion mechanics and presets from popular games such as Super Mario 64, Donkey Kong 64, and Banjo Kazooie, as well as our own game called Boss'n Run. 
    The study involves the creation of a dynamic player character using Blender and Unreal Engine 5. 
    The movement mechanics are categorized and polished with animations, sound effects, and particle effects. 
    The paper presents an evaluation method that involves collecting data from the games, comparing motion parameters, and obtaining feedback from 14 participants who played obstacle courses with different movement presets. 
    Results show that Boss'n Run and Banjo Kazooie received positive feedback, while the Super Mario 64 preset performed poorly. 
    Future work includes analyzing more movement options and implementing additional movement states, such as swimming, to further investigate the flow of movement in 3D platformers.
\end{abstract}

\begin{IEEEkeywords}
Movement Flow, 3D Jump'n Run, Procedural Content Generation
\end{IEEEkeywords}

%%%%%%%%%%%%%%%%%%%%%%%%%%%%%%%%%%%%%%%%%%%%%%%%%%%%%%%%%%%%%%%%%%%%%%%%%%%%%%%%
\section{Introduction}

The Super Mario Bros series is one of the best-known and most popular platformers on the market.
But why is it so much fun to play?
In his book \textit{Game Feel}, Steve Swink describes this phenomenon as game flow with the words:
%Steve Swink defines game flow in his book \textit{Game Feel} as:

\begin{quote}
    Real-time control of virtual objects in a simulated space, emphasizing interactions through polish \cite{swink2008game}.
\end{quote}

Game flow is therefore based on three key elements:

\begin{enumerate}
    \item Real-Time Control, e.g., the input processing of the player entity or entities.
    \item Spatial Simulation, e.g., the level design, especially the arrangement of collisions and response systems.
    \item Polish, e.g., sound effects, animations, and particle effects played when interacting with the world
\end{enumerate}

With respect to 3D Jump'n Runs, this means: Reasonable control and response of our player character, a game world that matches the player's movement parameters, e.g., his jump height and distance, and audio-visual cues that emphasize the player character's interactions in the world.
To investigate and analyze movement flow, we first developed a broad set of movement mechanics typically found in 3D Jump'n Runs.
To choose the right parameters for these mechanics, we analyze three influential games in the genre: Super Mario 64 \cite{SuperMario64}, Banjo Kazooie \cite{BanjoKazooie} and Donkey Kong 64 \cite{DonkeyKong64}.
Based on this analysis we added our own configuration called Boss'n Run.
To evaluate these four configurations, we procedural generated a level based on each indivdual set of movement paramters. 
! MARC PLEASE HELP! INTRODUCTIONS ARE SCARY! 
Maybe 2-3 sentences about PCG and how they are restrained in regards of movement parameters as well as they are always solvable!

\section{Related Work}
\label{Sec:RelatedWork}

Designing the optimal movement mechanics for a game is a elaborate task. 
Choosing the right parameters for your characters jump and run invovles an iterative process of gathering feedback and fine-tuning the parameters.
Therefore we propose an analytical approach:
All movement parameters are compared and visualized as graphs.
A jump can be defined as a three dimensional curve with the time and distance in x- and y-plane and height on the z-axis.
A similar definiton can be made for the movement again with the time and distance in x- and y-plane but this time the velocity on the z-axis \cite{gameMakerTool}.
These parameters are tightly coupled to the sourunding game world, which makes it hard to explore vastly different values as it involves changing your level layout.
! MARC PLEASE HELP! 
PCG comes to rescue since we can focus solely on analyzing different movement parameters and talk about freedom if you are at it ;)

\section{Methodology}
\label{Sec:Methodology}

We designed and implemented a 3D Jump'n Run with a broad set of movement mechanics in \textit{Unreal Engine 5}\footnote{\url{https://www.unrealengine.com/en-US}} 
as well as a procedural level generator with \textit{Blender's} Geometry Nodes\footnote{\url{https://www.blender.org/}}.

\subsection{Prerequisites}

To create a dynamic player character, we used \textit{Blender}\footnote{\url{https://www.blender.org/}}, a 3D modeling software, to model and design our character.
After this process, we obtained a model with a high density of vertices, which is unsuitable for use in a real-time environment because it is too computationally intensive.
Therefore, we need to retopologize the model to create a more uniform and less dense version (up to 100,000 vertices is fine).
We rigged the character using \textit{accuRig}\footnote{\url{https://actorcore.reallusion.com/auto-rig}}, an automatic rigging tool.
Rigging describes the process of creating a skeleton for a 3D model so that it can be more easily animated \cite{rigging}.
Instead of creating our own animations for our character's various movement options, which would be incredibly time consuming, we used Unreal's retargeting mechanic \cite{retarget}.
IK Retargeting allows us to apply animations developed for other characters to our character.
With this, we now had all the tools we needed to create a compelling player character.

\subsection{Movement mechanics}

An overview of all the movement mechanics we have implemented in Unreal Engine 5 can be seen in figure \ref{fig:mechanics}.

\begin{figure}[!ht]
    \centering
    \includegraphics[width=0.8\linewidth]{images/mechanics.png} % Adjust the width as needed
    \caption{Hierarchically structured taxonomy of all movement states that the player character can enter in the game world.}
    \label{fig:mechanics}
\end{figure}

The most basic function is the idle or walk state, which the character enters while moving.
From there, the player can perform the triple jump mechanic.
When the player presses the jump button, the character initially enters the normal jump state. 
After the character lands, the player has a window of $\textit{0.25s}$ to press the jump button again.
This allows the character to transition to the double jump state and then to the triple jump state.
These three jump states differ only in the speed at which the character is flung into the air. \\
If the character is in the idle/walking state and holds down the sprint button, he can run faster and switch to the sprint state.
Now when you perform a jump, it becomes a long jump.
The long jump catapults the character into the air with the least amount of force, but because of the faster ground movement, the player can cover greater distances with this type of jump. \\
When the character jumps against a wall, he enters the wall slide state. 
This mechanic slows the character's gravity drag in the air and near a wall.
From there, the player can press the climb button to enter the climbing state.
In the climbing state, gravity does not affect the character, and he can move freely on the surface of a climbable game object.
In each frame, the climbing logic performs a collision check to see if a climbable surface is still within reach of the character.
The player can move the character away from a valid query, resulting in an automatic exit from the climbing state, i.e., the player climbs up a wall and at the end the character automatically performs a climb-up animation and exits the climbing state.
While sliding or climbing the wall, the player can also press the jump button, which triggers the wall jump.
When wall jumping, the character is launched not only with a velocity upwards, but also with a velocity to the side, away from the wall.
By default, the character is launched in the direction of the normal vector of the queried surface.
The mechanics of the wall jump gives the player further control by allowing them to control the launch angle with respect to the wall, i.e. they can even make the character jump along the wall.
This concludes the rough overview of all the movement mechanics we have implemented.

\subsection{Polish}

To highlight the different states and their transitions, we added animations and blendspaces to all of them.
For example, the transition from idle to walking and sprinting is done with a blendspace that automatically transitions between the three different animations based on a value, in this case the character's movement speed.
An animation graph (in Unreal it is called Animation Blueprint) takes care of the transition between the different animations/blendspaces based on transition rules, such as whether the character is in the air.
These game logic states are injected by the character's blueprint.
To make the character's interaction with the world more immersive, we added particle effects for the different states.
For his steps when he lands on the ground and while he slides down a wall, we added a dust cloud particle effect.
While he climbs, we added a pebble particle effect.
All particle effects get stronger the more force is applied to the character, so when he slides down a wall, the dust cloud gets bigger the faster the character slides down.
Last but not least, we have added different sound effects for the different jumps or steps/climbing.

\section{Evaluation}
\label{Sec:Evaluation}

To evaluate the flow of movement for the mechanics implemented so far, we need to look at the parameters that lie at the base of these mechanics.
For example, the walking mechanic of a player character is defined by \textit{a)} its acceleration \textit{b)} its maximum speed that can be reached and \textit{c)} the deceleration to come back to a stop.
To get a good feel for the different parameters, we studied other famous games that rely heavily on their movement mechanics and collected their data.
Then we compared these different motion parameter presets by plotting them against each other.
Finally, we evaluate the motion flow based on the feedback from the players that we collected in a play session.

\subsection{Gathering data from other games}

Since some movement mechanics are inspired by Super Mario 64, we chose it as a comparison game.
As well as other 3D platformers/adventures released at a similar time,
in particular Donkey Kong 64 \cite{DonkeyKong64} and Banjo Kazooie \cite{BanjoKazooie}.
All three games rely heavily on the player character's movement mechanics for their gameplay.
This makes them ideal candidates.
To gather information on movement parameters, such as maximum running speed or jump height, we recorded video footage from the respective game.
We then analyzed it frame by frame using Shotcut\footnote{\url{https://shotcut.org/}}, a free, open-source, cross-platform video editor. 
For maximum running speed, we captured footage of the character running at full speed for a given estimated distance and time.
We then calculated acceleration/deceleration by analyzing video footage of the character starting and stopping.
In Super Mario 64, for example, the acceleration phase is indicated by a particle system, some dust clouds that stop spawning once the character has fully accelerated.
Now that we know the maximum running speed and the time it takes the character to fully accelerate/decelerate, we can calculate the acceleration and deceleration using the formula \ref{eq:a}.

\begin{equation}
    a = \frac{\Delta v}{\Delta t}
    \label{eq:a}
\end{equation}

For the jump height and gravity, we included material of the character jumping/falling over a certain estimated distance and tracking his time in the air.
If we use the formula for projectile motion \cite{gdc} and rearrange it, we can calculate gravity with the formula \ref{eq:g}.

\begin{equation}
    g = -\frac{2h}{t^2}
    \label{eq:g}
\end{equation}

If we know the gravity and the estimated jump height, we can calculate the jump velocity of the player character using the formula \ref{eq:vj}.

\begin{equation}
    v_j = \frac{h - \frac{1}{2}gt^2}{t}
    \label{eq:vj}
\end{equation}

All of the data collected is influenced by estimates to some degree, since none of these games have a reference object in metric units, such as a square meter cube.
Game Maker's Toolkit, for example, collects data from 2D platform games and uses the height and pixels of the player characters as distance metrics \cite{gameMakerTool}.
Thus, to obtain metric-style motion parameters, estimates must be made.

\subsection{Comparing movement parameters}

After collecting and calculating all the movement parameters of the different games, we present them in two different diagrams:

\begin{enumerate}
    \item Movement over time and distance, e.g. visualization of acceleration and deceleration phase by adding constant time window of \textit{0.5s} in which the character moves at full speed.
    \item Jump Height over time and distance, e.g. visualization of the character's jump at full movement speed.
\end{enumerate}

Looking at the figure \ref{fig:walk_mov}, we can gain some useful insights between the movement parameters of the different games.

\begin{figure}[!ht]
    \centering
    \includegraphics[width=0.45\textwidth]{images/WalkMov.png} % Adjust the width as needed
    \caption{Change of the player character's movement in three phases: \\
    (a) the acceleration phase, starting from zero velocity (idle state) to max walk velocity (walk state), 
    (b) the constant phase, moving at full walk speed for a fixed amount of time (0.5s), and 
    (c) the deceleration phase, to get from full walk speed (walk state) back to zero velocity (idle state). \\
    Velocity is plotted on the z-axis over time on the x-axis and distance on the y-axis needed to go through the three phases.}
    \label{fig:walk_mov}
\end{figure}

\textit{Boss'n Run} has the highest speed, acceleration and deceleration of the 4 games.
\textit{Super Mario 64} has the lowest speed, acceleration and deceleration, also it takes the longest amount of time to start running and come to a stop again.
\textit{Donkey Kong 64} takes the longest distance traveled to start running and come to a stop again. 
\textit{BanjoKazooie} takes the least time and the least distance to run and come to a stop again.
All movement parameters are recorded and compared in the table \ref{tab:insightII}.

\begin{table}[htbp]
    \caption{Movement parameters of the games. For speed, acceleration, and deceleration, highest values are marked in red and lowest values in blue. For time and distance to reach maximum speed, lowest values are marked in red and highest values in blue.}
    \label{tab:insightII}
    \centering
    \begin{tabularx}{\linewidth}{l*{4}{>{\centering\arraybackslash}X}}
    \toprule
     & \textbf{Boss'n Run} & \textbf{Super Mario 64} & \textbf{Banjo Kazooie} & \textbf{Donkey Kong 64} \\
    \midrule
    \textbf{Velocity ($\frac{\mathrm{m}}{\mathrm{s}}$)} & \textcolor{red}{6} & \textcolor{blue}{4} & 4.5 & 5.4 \\
    \textbf{Acceleration ($\frac{\mathrm{m}}{\mathrm{s^2}}$)} & \textcolor{red}{20} & \textcolor{blue}{8} & 18 & 12.8 \\
    \textbf{Deceleration ($\frac{\mathrm{m}}{\mathrm{s^2}}$)} & \textcolor{red}{20} & \textcolor{blue}{8} & 18 & 12.8 \\
    \textbf{Time (s)} & 1.1 & \textcolor{blue}{1.5} & \textcolor{red}{1.0} & 1.35 \\
    \textbf{Distance (m)} & 4.8 & 4 & \textcolor{red}{3.38} & \textcolor{blue}{4.98} \\
    \bottomrule
    \end{tabularx}
\end{table}

Looking at the figure \ref{fig:jump} some useful information can be gained about the jump parameters of the 4 different games.

\begin{figure}[!ht]
    \centering
    \includegraphics[width=0.45\textwidth]{images/DJump.png}
    \caption{Change of the player character's jump height (z-axis) at full walk speed, performing a normal jump (jump state), over the time in the air (x-axis) and the distance covered with the jump (y-axis).}
    \label{fig:jump}
\end{figure}

\textit{Boss'n Run} still has the highest ground speed and therefore can travel the farthest distance with its jump.
\textit{Super Mario 64} has the highest gravity of all the games and also the highest jump speed, resulting in the greatest jump height. Nevertheless, its ground speed is the lowest, resulting in the shortest jump distance.
\textit{Donkey Kong 64} has the lowest jump height and the least time in the air of all.
\textit{Banjo Kazooie} has the lowest gravity and jump speed, but the longest time in the air of all.
All jumping paramters are recorded and compared in table \ref{tab:insightIII}.

\begin{table}[htbp]
    \caption{Default jump parameters of the 4 games Boss'n Run, Super Mario 64, Donkey Kong 64, and Banjo Kazooie. For all parameters, the highest values are marked in red and the lowest in blue.}
    \label{tab:insightIII}
    \centering
    \begin{tabularx}{\linewidth}{l*{4}{>{\centering\arraybackslash}X}}
    \toprule
     & \textbf{Boss'n Run} & \textbf{Super Mario 64} & \textbf{Banjo Kazooie} & \textbf{Donkey Kong 64} \\
    \midrule
    \textbf{Gravity ($\frac{\mathrm{m}}{\mathrm{s^2}}$)} & -18.6 & \textcolor{red}{-20.7} & \textcolor{blue}{-14.0} & -20.2 \\
    \textbf{Ground Velocity ($\frac{\mathrm{m}}{\mathrm{s}}$)} & \textcolor{red}{6} & \textcolor{blue}{4} & 4.5 & 5.4 \\
    \textbf{Jump Velocity ($\frac{\mathrm{m}}{\mathrm{s}}$)} & 8.7 & \textcolor{red}{9.3} & \textcolor{blue}{7.3} & 8.2 \\
    \textbf{Height (m)} & 2.03 & \textcolor{red}{2.09} & 1.84 & \textcolor{blue}{1.66} \\
    \textbf{Time (s)} & 0.93 & 0.9 & \textcolor{red}{1.0} & \textcolor{blue}{0.81} \\
    \textbf{Distance (m)} & \textcolor{red}{5.6} & \textcolor{blue}{3.6} & 4.5 & 4.4 \\
    \bottomrule
    \end{tabularx}
\end{table}

\subsection{Feedback from players}

Comparing movement parameters on a mathematical level already leads to interesting insights, e.g. for level design.
Nevertheless, we can not evaluate any of these values about their movement flow.
Therefore, we prepared a pilot study with 14 participants.
The idea is to evaluate the different movement presets of each game by having the participants play different obstacle courses based on these specifications.
To make this possible, we abstracted all the movement parameters into a data asset, a simple asset that stores data related to a specific system in an instance \cite{dataAsset}.
This allowed us to create data assets for each movement preset and set those parameters at the beginning of a level.
Then we used the same data asset instance to generate an obstacle course based on the movement parameters, e.g., it was necessary that the platform distances and heights were reachable with the current default.
Participants were asked three questions:

\begin{enumerate}
    \item How do you like the jump mechanics? (Rated from 1 (bad) to 5 (good))
    \item How is the overall Game Flow for you? (Rated from 1 (poor) to 5 (fluid))
    \item How difficult was the level for you? (Rated from 1 (easy) to 5 (difficult))
\end{enumerate}

We have summarized the results in the figure \ref{fig:eval}.
Boss'n Run and Banjo Kazooie had the best jumping feel of the 4 games, while Super Mario 64 had the worst.
Banjo Kazooie had the best overall gameplay, with Boss'n Run close behind and Super Mario 64 in last place.
Of all the levels, the Super Mario 64 level was felt to be the hardest and Banjo Kazooie the easiest.

\begin{figure}[!ht]
    \caption{The results of the feedback from the 14 participants who played different obstacle courses based on the 4 movement specifications: Boss'n Run (BnR), Super Mario 64 (SM64), Donkey Kong 64 (DK64), and Banjo Kazooie (BK). This was done to determine how good the jumping mechanics felt, how the overall flow of the game felt, and how difficult the obstacle course was perceived to be.}
    \centering
    \includegraphics[width=0.45\textwidth]{images/game_ratings}
    \label{fig:eval}
\end{figure}


One possible explanation for why Banjo Kazooie was perceived very well overall and Super Mario 64 poorly is the massive difference in gravity and resulting time in the air.
The longer time spent in the air in the Banjo Kazooie preset allowed players to easily adjust their jump direction, making precise jumps easier and less frustrating.
Several participants stated that Super Mario 64's preset character moved sluggishly compared to the other presets, which reduced gameplay enjoyment.
This sluggish behavior can be explained by the slow acceleration/deceleration parameters.
Other participants stated that regardless of the preset, everyone had an interesting gameplay experience and that it was just a matter of habit.


\section{Conclusion \& Future Work}
\label{Sec:ConcFuture}

We examined the movement flow by looking at all the parameters that make it up.
To do this, we created and implemented a player character with different movement options in Unreal Engine 5.
We evaluated the movement flow on an objective level by collecting and comparing the movement parameters between our own preset Boss'n Run and three famous games, namely Super Mario 64, Donkey Kong 64 and Banjo Kazooie.
And on the subjective level, we had 14 participants play different levels based on the 4 motion presets of these games.
Based on the current results, we can say that the Banjo Kazooie preset is the most promising, followed closely by our own Boss'n Run preset.
The Super Mario 64 preset performed the worst. \\
As mentioned in \ref{Sec:Methodology}, we implemented more movement options than we examined here. 
In a future work, it would be interesting to analyze and compare all the different jump types, as well as implement other motion states, such as swimming, which are common in games.

\section*{Acknowledgement}

A special thanks goes to Marc Mussmann who implemented a procedural obstacle generator.
This allowed us to create a small game that we hope the participants enjoyed.

% \addtolength{\textheight}{-12cm}   % This command serves to balance the column lengths
                                  % on the last page of the document manually. It shortens
                                  % the textheight of the last page by a suitable amount.
                                  % This command does not take effect until the next page
                                  % so it should come on the page before the last. Make
                                  % sure that you do not shorten the textheight too much.

%%%%%%%%%%%%%%%%%%%%%%%%%%%%%%%%%%%%%%%%%%%%%%%%%%%%%%%%%%%%%%%%%%%%%%%%%%%%%%%%



%%%%%%%%%%%%%%%%%%%%%%%%%%%%%%%%%%%%%%%%%%%%%%%%%%%%%%%%%%%%%%%%%%%%%%%%%%%%%%%%



%%%%%%%%%%%%%%%%%%%%%%%%%%%%%%%%%%%%%%%%%%%%%%%%%%%%%%%%%%%%%%%%%%%%%%%%%%%%%%%%


\bibliographystyle{IEEEtran}
\bibliography{literature.bib}

\end{document}
